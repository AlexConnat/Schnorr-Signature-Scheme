\documentclass{article}
\usepackage[utf8]{inputenc}

\title{Overview of the\\Schnorr Signature Scheme}
\author{Alexandre Connat}
\date{January, 2017}

\usepackage{natbib}
\usepackage{graphicx}
\usepackage{amsmath}
\usepackage{amssymb}

\usepackage{titling}

\setlength{\droptitle}{-10em} % To remove the very large margin on top of page


\begin{document}

\maketitle





\section{Introduction}

The Schnorr Signature Scheme, is a very simple cryptographic signature scheme, based on the hardness of the discrete logarithm problem. It is used notably in the Bictoin Protocol. \\
In this document we used the multiplicative notation to describe all the operations performed on the group. But if we work in additive groups, like Elliptic Curves, we could easily draw a parallel with the additive notation.  \\

\noindent
The goal of this scheme is for a sender to be able to sign a message $M$ with a signature $S$. \\
The sender wants to prove its identity and the authenticity of the message by sending the message $M$ alongside with the signature $S$, over a insecure channel. \\
The receiver should be able to verify the signature with quasi-certainty.







\section{Protocol}

Both sender and receiver must agree on a group to perform all the following operations. In this document, we'll focus on the group $G = Z_p^*$ with $p$ a big prime number. \\

\noindent
We assume the sender has generated a key-pair (Private Key, Public Key) = ($x$, $y$) with $y = g^x$ mod $p$.
We assume the receiver is able to access the Public Key $y$ of the sender.


\subsection{Sender-Side:}

Compute : \

\begin{itemize}

	\item $t = g^r$, \ where $r \in Z_p^*$ random

	\item $e =$ Hash($M \| t$), \ where Hash() is any hash function (Sha1, Sha256, ...) and $\|$ is the 'append' sign

	\item $s = r - x e$

\end{itemize}

$\Rightarrow$ The signature is $S = (s, e)$


\subsection{Receiver-Side:}

Compute : \

\begin{itemize}

	\item $l = g^s y^e$, \ with $s$ and $e$ from $S$, and $y$ the Public Key of the sender

	\item $e_v =$ Hash($M \| l$)

\end{itemize}

$\Rightarrow$ The signature is considered as :
	$
    \left\{
        \begin{array}{ll}
        	\textsc{Valid} \ \ , \  \text{if } \ e_v = e \\
        	\textsc{Invalid} \ \ , \  \text{if } \ e_v \neq e
        \end{array}
    \right.
   	$







\section{Proof - Correctness}

If we consider a valid key-pair $(x,y)$, with $y = g^x$, we have :

$$
\begin{align*}
l &= g^s \cdot y^e \\
  &= g^{r - xe} \cdot (g^x)^e \\
  &= g^r \cdot g^{-xe} \cdot g^{xe} \\
  &= g^r \\
  &= t
\end{align*}
$$

\noindent
Hence, we have Hash($M \| l$) $\equiv$ Hash($M \| t$) \

\noindent
Hence, we have $e_v \equiv e$





\section{Error Probability}

\begin{itemize}

	\item if $e_v \neq e$, signature is $\textsc{Invalid}$ with probability $1$.

	\item if $e_v = e$, signature is $\textsc{Valid}$ with probability $1 - \epsilon$. \ ($\epsilon \approx 0$ for large $p$)

\end{itemize}








\section{Proof - Error Probability}

Assume the key-pair is invalid. Let's define a random $x' \neq x$, s.t: $y \neq g^{x'}$  \\

\noindent
We will prove that there are two possibilities to forge a signature : \\
Either find the exact $x' = x$, which is the real Private Key (with probability $\frac{1}{|G|} = \frac{1}{p-1}$) or, by "chance" $e =$ Hash($M \| t$) is a multiple of $p$, and its reduction modulo $p$ is $0$ (with probability $\frac{1}{p}$). \\

We receive a forged signature : 
$S = (s', e) = (r - x'e, e)$ \\
\indent
We compute :
$l = g^{s'} \cdot y^e = g^r \cdot g^{-x'e} \cdot g^{xe} = g^r \cdot g^{e(x - x')} \\


$$
\begin{align*}
\Rightarrow Pr(l = g^r) = Pr(g^{e(x - x')} = 1) &= Pr(e = 0) + Pr(x = x') \\
            &= \frac{1}{p} + \frac{1}{p-1} \approx \frac{2}{p} \underset{p\to +\infty}{\longrightarrow} 0
\end{align*}
$$






\section{Implemented Protocol}

To write the actual code, we worked with an Elliptic Curve $E$, instead of $\mathbb{Z}_p$ and the protocol we used to sign and verify messages slightly differs from what we explained in Part 2. \\

\noindent
We worked with the curve $Ed25199$, defined over the prime field $\mathbb{F}_q$, of prime order $q = 2^{255} - 19$ and a base point $G = (x, -4/5)$ \\ with $\ell = 2^{252} + 27742317777372353535851937790883648493$ the prime order of the base point. \\

\noindent
We used Sha256 as our Hash Function, which outputs the result as a 256-bit string, which is further reduced modulo $q$, to be used as a scalar value in the group. \\

\noindent
The key-pair is (Private Key, Public Key) = ($x$, $Y$), with $Y = x \cdot G$, the point obtained by multiplying the base point $G$ by the scalar $x$.



\subsection{Sender-Side}

Compute : \

\begin{itemize}

    \item $R = k \cdot G$ \ , \ $k \in \mathbb{Z}_\ell^*$ random
    
    \item $e =$ Hash($M \| R$) \ , \ $M$ the message, and "Hash" the Sha256 function

    \item $s = k + xe$ \ , \ $x$ the private key of the sender

\end{itemize}

$\Rightarrow$ The signature is $S = (R, s)$



\subsection{Receiver-Side}

Compute : \

\begin{itemize}

  \item $e =$ Hash($M \| R$) \ , \ $M$ the message, and $R$ from the signature $S$

  \item $sg_v = R + e \cdot Y$ \ , \ $Y$ the public key of the sender

  \item $sg = s \cdot G$ \ , \ $s$ from the signature $S$

\end{itemize}

$\Rightarrow$ The signature is considered as :
	$
    \left\{
        \begin{array}{ll}
        	\textsc{Valid} \ \ , \  \text{if } \ sg_v = sg \\
        	\textsc{Invalid} \ \ , \  \text{if } \ sg_v \neq sg
        \end{array}
    \right.
  $



\section{Implementation}

We based our implementation on the Crypto library of the DEDIS laboratory (www.github.com/dedis/crypto) \\

\noindent
We used the cryptographic suite \verb|ed25519.NewAES128SHA256Ed25519()| \\

\noindent
Keys are generated using the method \verb|config.NewKeyPair()| \\
\verb|keypair.Secret| is of type \verb|abstract.Scalar| \\
\verb|keypair.Public| is of type \verb|abstract.Point| \\


We provide 2 public methods to the user programmer :

\begin{itemize}

    \item \verb|SignMessage(m, x)|
    which takes a message \verb|m| (of type \verb|string|), and a private key \verb|x| (of type \verb|abstract.Scalar|) as input, and outputs a signature \verb|S| (of type \verb|Signature|).

    \item \verb|VerifySignature(m, S, Y)|
    which takes a message \verb|m| (of type \verb|string|), a signature \verb|S| (of type \verb|Signature|), and a public key \verb|Y| (of type \verb|abstract.Point|) as input, and outputs a boolean value, representing whether the given signature is a $\textsc{Valid}$ signature (\verb|true|) or an $\textsc{Invalid}$ signature (\verb|false|) for the given message, with this public key. \\
    (Alternatively, you can use the function \verb|Verify(m, Y)| directly on the Signature $S$ itself)

\end{itemize}

\noindent
The signature itself is held in a custom type struct :
\begin{verbatim}
type Signature struct {
    R abstract.Point
    s abstract.Scalar
}    
\end{verbatim}





\section{Testing}

We tested the code with : \\
- a valid key-pair, a valid signature, a valid message \\
- valid message and signature, invalid keypair \\
- valid keypair and signature, invalid message to verify \\
+ Marshaling/Unmarshaling

ETC.... \\

We tested corner cases and errors launching e.g Empty signature, invalid key-pair numbers, empty message to sign???? \\



\end{document}
